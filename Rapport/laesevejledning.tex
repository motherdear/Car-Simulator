\documentclass[rapport.tex]{subfiles}

\begin{document}

\section{Læsevejledning}

Kildehenvisninger i rapporten sker med fodnoter, hvori to klammer omkranser et nummer, som refererer til litteraturlisten -- f.eks. \cite{REGULERINGSTEKNIK}, som refererer til bogen \emph{Reguleringsteknik}.

Det skal bemærkes, at der i rapporten er blevet anvendt punktum som decimalseparator.

Til sidst i rapporten vil man kunne finde en nomenklaturliste, der giver en oversigt over fagudtryk og symboler, en række bilag og en litteraturliste.

Sammen med rapporten er vedlagt en CD. Herpå findes der en elektronisk kopi af rapporten, filer med kildekode, der vil gøre det muligt at fortsætte på projektet og samtlige datablade på de anvendte elementer.

\end{document}
